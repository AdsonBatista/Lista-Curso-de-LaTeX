%----------------------------------------------------------------
%Utilize compilação XELaTeX para que tudo saia como deveria sair.
%----------------------------------------------------------------
\documentclass[12pt]{article}
%%Metodos de entrada
\usepackage[brazil]{babel}
\usepackage[T1]{fontenc}
%\usepackage[utf8]{inputenc}
%%Definiçoes de margem e roda pé... coloque showframe em [] e veja os limites
\usepackage[a4paper,left=2cm,right=2cm,top=3.5cm,bottom=3.5cm,footskip=2cm,headheight=3cm]{geometry}
\usepackage{fancyhdr} 			%Pacote para cabeçalho
\usepackage{lastpage}			%pacote para verificar numero de paginas do documento
\usepackage{longtable}			%Pacote para criar tabelas que quebram as paginas
\usepackage{graphicx}			%Pacote para importação de imagens
\usepackage{xcolor}				%pacote de cores
\definecolor{IEEEBlue}{RGB}{0, 98, 155} %define a cor Azul IEEE

%todo: Pesquisar como colocar propriedades dos documentos em XELaTeX.

%\usepackage[pdfauthor=Adson Batista, pdftitle={Documentos UFABC}, pdfsubject={Documentos UFABC, Lista presença}, pdfkeywords={Cursos 2017,IEEE UFABC, Documentos}, pdfproducer=``Diretoria Ramo 2017'', pdfcreator=pdflatex]{hyperref}  %pacote para link e email + METADATA
%%%%%Fonte Verdana%%%

%%%Cria box e define cordenadas para numeração de pagina
\usepackage[pscoord]{eso-pic}% The zero point of the coordinate systemis the lower left corner of the page (the default).

\newcommand{\placetextbox}[3]{% \placetextbox{<horizontal pos>}{<vertical pos>}{<stuff>}
  \setbox0=\hbox{#3}% Put <stuff> in a box
  \AddToShipoutPictureFG*{% Add <stuff> to current page foreground
    \put(\LenToUnit{#1\paperwidth},\LenToUnit{#2\paperheight}){\vtop{{\null}\makebox[0pt][c]{#3}}}%
  }%
}%

% Definição de fonte
\usepackage{fontspec}
\defaultfontfeatures{Mapping=text-text}
\setsansfont{Verdana}
\renewcommand*{\familydefault}{\sfdefault}

%% Imagem do Cabeçalho
\fancyhead[OR]{
	\includegraphics[height=2cm]{imagens/logo_ufabc.pdf}
	\vspace{-.3cm}
}

%% Linha do Cabeçalho
\renewcommand{\headrulewidth}{0.5pt}% 2pt header rule
\renewcommand{\headrule}{\hbox to\headwidth{%
\color{IEEEBlue}\leaders\hrule height \headrulewidth\hfill}}

%% Texto do Rodapé
\fancyfoot[OC]{\begin{tabular}{c{}}Student Branch IEEE UFABC \\ Avenida dos Estados, 5001 - Sala 009-B -  Bloco A\ \\ CEP 09210-170 - Bangu - Santo Andre - SP - Brasil\\ contato@ieeeufabc.org\end{tabular}}
\rfoot{}
%% Linha do Rodapé
\renewcommand{\footrulewidth}{0.5pt}% Default \footrulewidth is 0pt
\renewcommand{\footrule}{\hbox to\headwidth{%
\color{IEEEBlue}\leaders\hrule height \headrulewidth\hfill}}

%Define uma caixa em background
\usepackage[color=black,opacity=1,angle=0,scale=1]{background}
\backgroundsetup{
  contents={\placetextbox{.845}{.91}{Pg. \thepage \hspace{1pt} de \pageref{LastPage}}%
	\vspace{-10cm}
	}
}
%% Inicio do documento
\begin{document}
	\pagestyle{fancy} %define tipo de pagina como a com o logo do IEEE
%Titulo do documento vai aqui
	\begin{center}
		\textcolor{IEEEBlue}{\LARGE Lista de presenca curso de \LaTeX}
	\end{center}
	\vspace{1.5cm}
%Tabela com as pessoas - neste caso que são dois dias foram 3 colunas.
	\begin{longtable}{lll}
%Estas linhas tem o campo para identificar a pessoa responsavel pelo evento e o dia que ela foi responsável.
										  			& \rule{4cm}{.1mm} 			&\rule{4cm}{.1mm}			\\[5pt]
										  			& \textbf{Responsavel no}	&\textbf{Responsavel no} 	\\
										  			& \textbf{dia 4 de Abril}	&\textbf{dia 6 de Abril} 	\\[25pt]
		\textbf{Nome}    				  			& \textbf{Assinatura} 		&\textbf{Assinatura} 		\\[15pt]
%inicio dos nomes \hrulefill preenche o restante do espaço com uma linha
		Ana Beatriz Araujo Carballeira 	 \hrulefill & \rule{4cm}{.1mm} 			&\rule{4cm}{.1mm}			\\[15pt]
		Camila do Amaral  Sass 			 \hrulefill & \rule{4cm}{.1mm} 			&\rule{4cm}{.1mm}			\\[15pt]
		Eddie Weiss 					 \hrulefill & \rule{4cm}{.1mm} 			&\rule{4cm}{.1mm}			\\[15pt]
		Gabriela Bertoni dos Santos      \hrulefill & \rule{4cm}{.1mm} 			&\rule{4cm}{.1mm}			\\[15pt]
		Guilherme Ribeiro Portugal       \hrulefill & \rule{4cm}{.1mm} 			&\rule{4cm}{.1mm}			\\[15pt]
		Leonardo Andrade Castro          \hrulefill & \rule{4cm}{.1mm} 			&\rule{4cm}{.1mm}			\\[15pt]
		Matheus Tulio Pereira da Cruz    \hrulefill & \rule{4cm}{.1mm} 			&\rule{4cm}{.1mm}			\\[15pt]
		Pedro Gabriel Lourenço           \hrulefill & \rule{4cm}{.1mm} 			&\rule{4cm}{.1mm}			\\[15pt]
		Raquel da Silva Barros           \hrulefill & \rule{4cm}{.1mm} 			&\rule{4cm}{.1mm}			\\[15pt]
		Vitoria Garcia Alvarez           \hrulefill & \rule{4cm}{.1mm} 			&\rule{4cm}{.1mm}			\\[15pt]
		Wellington do Nascimento Pinheiro\hrulefill & \rule{4cm}{.1mm} 			&\rule{4cm}{.1mm}			\\[15pt]
		\rule{7cm}{.1mm} 				 			& \rule{4cm}{.1mm} 			&\rule{4cm}{.1mm}			\\[15pt]
		\rule{7cm}{.1mm} 				 			& \rule{4cm}{.1mm} 			&\rule{4cm}{.1mm}			\\[15pt]
		\rule{7cm}{.1mm} 				 			& \rule{4cm}{.1mm} 			&\rule{4cm}{.1mm}			\\[15pt]
		\rule{7cm}{.1mm} 				 			& \rule{4cm}{.1mm} 			&\rule{4cm}{.1mm}			\\[15pt]
		\rule{7cm}{.1mm} 				 			& \rule{4cm}{.1mm} 			&\rule{4cm}{.1mm}			\\[15pt]
		\rule{7cm}{.1mm} 				 			& \rule{4cm}{.1mm} 			&\rule{4cm}{.1mm}			\\[15pt]
		\rule{7cm}{.1mm} 				 			& \rule{4cm}{.1mm} 			&\rule{4cm}{.1mm}			\\[15pt]
		\rule{7cm}{.1mm} 				 			& \rule{4cm}{.1mm} 			&\rule{4cm}{.1mm}			\\[15pt]
		\rule{7cm}{.1mm} 				 			& \rule{4cm}{.1mm} 			&\rule{4cm}{.1mm}			\\[15pt]
		\rule{7cm}{.1mm} 				 			& \rule{4cm}{.1mm} 			&\rule{4cm}{.1mm}			\\[15pt]
		\rule{7cm}{.1mm} 				 			& \rule{4cm}{.1mm} 			&\rule{4cm}{.1mm}\\
	\end{longtable}

%Fim do documento
\end{document}
